\documentclass{elsarticle}
\usepackage[utf8]{inputenc}
\usepackage{kpfonts}
\usepackage[labelfont=bf]{caption}
\usepackage{graphicx}
\usepackage{subfig}
\usepackage{stmaryrd}
\usepackage[usenames, dvipsnames]{color}
\usepackage{esvect}
\usepackage{enumitem}
\captionsetup[figure]{name=Fig. ,labelsep=period}

\newcommand{\choice}{|}
\newdefinition{definition}{Definition}
\newtheorem{notation}{Notation}
\newtheorem{theorem}{Theorem}
\newdefinition{example}{Example}
\newdefinition{running_example}{Running example}

\begin{document}

\section{Objects definition}

Let $\mathcal{N}_{A},~\mathcal{N}_{T},~\mathcal{N}_{\delta},~\mathcal{N}_{c}$ be mutually exclusive finite sets of atomic, structure, state names, and compartments, respectively.

\begin{definition}{Signature}

\begin{itemize} 
\item Atomic signature $\Sigma_{\mathtt{A}} \subseteq \mathcal{N}_{A} \times 2^{\mathcal{N}_{\delta}}$ represents allowed atomic names with association to state names. 
\item Structure signature $\Sigma_{\mathtt{T}} \subseteq \mathcal{N}_{T} \times 2^{\mathcal{N}_{A}}$ represents allowed structure names with association to atomic names. 
\end{itemize}
\end{definition}

\begin{example}{Signature}\label{example:signature}

\begin{itemize}
\item An atomic signature $(S, \{u, p\})$, $(Q, \{a, i\})$
\item A structure signature $(KaiC, \{S, Q\})$
\end{itemize}
\end{example}

\noindent\rule{\textwidth}{1pt}

\begin{definition}{Atomic agent}

\begin{itemize}
\item Atomic agent $\mathtt{A}$ is a pair $(\mu, \delta)$ where $\mu \in \mathcal{N}_{A}$ is a name and $\delta \in \mathcal{N}_{\delta} \cup \{ \varepsilon \}$ is a state.

\item Equivalence: $\mathtt{A} \equiv \mathtt{A}' \Leftrightarrow \mu(\mathtt{A}) \equiv \mu(\mathtt{A}') \wedge \delta(\mathtt{A}) \equiv \delta(\mathtt{A}')$
\end{itemize}
\end{definition}

\begin{notation}
Let $\mathcal{A}$ be universe of all possible atomic agents.
\end{notation}

\begin{example}{Atomic agent}\label{example:atomic}

\begin{itemize}
\item An atomic agent $\mathtt{A}_1 = (S, u)$
\item An atomic agent $\mathtt{A}_2 = (Q, \varepsilon)$
\end{itemize}
\end{example}

Please note state $\varepsilon$ does not mean the agent is in \emph{none} state, but rather the state is unknown or not important.

\noindent\rule{\textwidth}{1pt}

\begin{definition}{Structure agent}

\begin{itemize}
\item Structure agent $\mathtt{T}$ is a pair $(\mu, \gamma)$ where $\mu \in \mathcal{N}_{T}$ is a name and $\gamma \subseteq \mathcal{A}$ is a set of atomic agents called partial composition.

\item Equivalence: $\mathtt{T} \equiv \mathtt{T}' \Leftrightarrow \mu(\mathtt{T}) \equiv \mu(\mathtt{T}') \wedge \forall \mathtt{A} \in \gamma(\mathtt{T}) \exists \mathtt{A}' \in \gamma(\mathtt{T}') : \mathtt{A} \equiv \mathtt{A}' \wedge \\ \wedge \forall \mathtt{A}' \in \gamma(\mathtt{T}') ~\exists \mathtt{A} \in \gamma(\mathtt{T}) : \mathtt{A}' \equiv \mathtt{A}$.
\end{itemize}
\end{definition}

\begin{notation}
Let $\mathcal{T}$ be universe of all possible structure agents.
\end{notation}

\begin{example}{Structure agent}\label{example:structure}

\begin{itemize}
\item A structure agent $\mathtt{T}_1 = (K, \textcolor{red}{\{}(S, p), (Q, i)\textcolor{red}{\}})$
\item A structure agent $\mathtt{T}_2 = (K, \textcolor{red}{\{}(Q, a)\textcolor{red}{\}})$
\end{itemize}
\end{example}

\noindent\rule{\textwidth}{1pt}

\begin{definition}{Complex}

\begin{itemize}
\item Complex $\mathtt{X}$ is a pair $(\mu, \mathtt{com})$ where $\mu \in (\mathcal{A} \cup \mathcal{T})^n$ is a sequence and $\mathtt{com} \in \mathcal{N}_{c}$ is a compartment.

\item Let $\mathtt{X}, \mathtt{X}'$ be complexes. We denote length of sequence $\mu$ as $n = \mu(\mathtt{X})$ and $m = \mu(\mathtt{X}')$. 

\item Equivalence: $\mathtt{X} \equiv \mathtt{X}' \Leftrightarrow n \equiv m \wedge \exists $ permutation $ \mu'(\mathtt{X}')$ of $\mu(\mathtt{X}')$ such that $\forall i~\in~[1,~n]: \mu(\mathtt{X})_i \equiv \mu'(\mathtt{X}')_i$.
\end{itemize}
\end{definition}

\begin{notation}
Let $\mathcal{X}$ be universe of all possible complexes.
\end{notation}


\begin{example}{Complex}\label{example:complex}

\begin{itemize}
\item A complex $\mathtt{X} = (\textcolor{green}{(}(K, \textcolor{red}{\{}(S, p), (Q, i)\textcolor{red}{\}}), (S, p)\textcolor{green}{)}, cytosol)$
\end{itemize}
\end{example}

\noindent\rule{\textwidth}{1pt}

\begin{definition}{Rule}

Rule $\mathtt{R}$ is a 5-tuple $(\chi, \omega, \mathtt{I}, \mathtt{map}, \mathtt{Indices})$ where:

\begin{itemize}
\item $\chi \in \mathcal{X}^n$ is a sequence of complexes,
\item $\omega \in (\mathcal{A} \cup \mathcal{T})^m$ is a sequence of atomic and structure agents,
\item $\mathtt{I} \in \{ 1, \ldots, n \}$ is an index determining start of right-hand-side,
\item $\mathtt{map} \in \mathbb{N}^m$ is an index map between $\omega$ and $\chi$,
\item $\mathtt{Indices} \subseteq (\{-\} \cup \mathbb{N})^2$ is an index map between agents from left- and right-hand side
\end{itemize}

where $n, m \in \mathbb{N}$.
\end{definition}

\begin{example}{Rule}\label{example:rule}

\begin{itemize}
\item 
$\chi = \begin{bmatrix}
(\textcolor{green}{(} (K, \textcolor{red}{\{}(S, u)\textcolor{red}{\}}), (B, \textcolor{red}{\emptyset}) \textcolor{green}{)}, cyt),\\

(\textcolor{green}{(} (C, \textcolor{red}{\emptyset}), (D, i) \textcolor{green}{)}, cyt),\\

(\textcolor{green}{(} (A, \varepsilon) \textcolor{green}{)}, cyt),\\

(\textcolor{green}{(} (K, \textcolor{red}{\{}(S, p)\textcolor{red}{\}}), (B, \textcolor{red}{\emptyset}), (C, \textcolor{red}{\emptyset}) \textcolor{green}{)}, cyt),\\

(\textcolor{green}{(} (D, a), (A, \varepsilon) \textcolor{green}{)}, cyt) 
\end{bmatrix}$

\item $\omega = \begin{bmatrix}
(K, \textcolor{red}{\{}(S, u)\textcolor{red}{\}}), (B, \textcolor{red}{\emptyset}), (C, \textcolor{red}{\emptyset}), \\
(D, i), (A, \varepsilon), (K, \textcolor{red}{\{}(S, p)\textcolor{red}{\}}), \\
(B, \textcolor{red}{\emptyset}), (C, \textcolor{red}{\emptyset}), (D, a), (A, \varepsilon)
\end{bmatrix}$

\item $\mathtt{I} = 3$
\item $\mathtt{map} = (2,4,5,8,10,11)$
\item $\mathtt{Indices} = [ (1,6) ; (2,7) ; (3,8) ; (4,9) ; (5,10) ; (-,11) ] $
\end{itemize}
\end{example}

\noindent\rule{\textwidth}{1pt}

\begin{definition}{Reaction}

Reaction $\mathtt{r}$ is a pair $(\mathtt{seq}, \mathtt{I})$ where:

\begin{itemize}
\item $\mathtt{seq} \in \mathcal{X}^n$ is a sequence of complexes,
\item $\mathtt{I} \in \{ 1, \ldots, n \}$ is an index determining start of right-hand-side
\end{itemize}

where $n, m \in \mathbb{N}$.
\end{definition}

\noindent\rule{\textwidth}{2pt}

\section{Syntax}

\begin{definition}
\textit{Grammar.}

\begin{center}
\begin{tabular}{ l l }
atomic expression & $\mathtt{a} ::= \mu\{s\} ~|~ ? \mu$\\
 & $\mu ::= n \in \mathcal{N}_{A}$ \\
 & $s ::= n \in \mathcal{N}_{\delta}$\\
 & \\
structure expression & $\mathtt{t} ::= \mu(\gamma) ~|~ \mu$\\
 & $\gamma ::= \mathtt{a}_1, \ldots, \mathtt{a}_m$ \\
 & $\mu ::= n \in \mathcal{N}_{T}$\\
 & \\
complex expression & $\Gamma ::= \alpha_1~.~\ldots~.~\alpha_k :: c$\\
 & $\alpha_i ::= \mathtt{a} ~|~ \mathtt{t}$\\
 & $c ::= n \in \mathcal{N}_{c}$\\
 & \\
rule & $\mathtt{R} ::= \Gamma_1 + \ldots + \Gamma_n \Rightarrow \Gamma_{n+1} + \ldots + \Gamma_m $
\end{tabular}

\end{center}
where $m,n \in \mathbb{N}_0 \wedge m + n \neq 0$ and $k \in \mathbb{N}$.
\end{definition}

\begin{example}{Syntax}\label{example:syntax}

In following goes examples of agents, complexes and rules given in previous examples:

\begin{itemize}
\item $\mathtt{A}_1$ is written as $S\{u\}$, $\mathtt{A}_2$ is written as $Q$ (Example~\ref{example:atomic}),

\item $\mathtt{T}_1$ is written as $K(S\{p\}, Q\{i\})$, $\mathtt{T}_2$ is written as $K(Q\{A\})$ (Example~\ref{example:structure}),

\item $\mathtt{X}$ is written as $K(S\{p\}, Q\{i\}).S\{p\}::cytosol$ (Example~\ref{example:complex}),

\item rule $\mathtt{R}$ is written as (Example~\ref{example:rule}):
\end{itemize}

\noindent {\small $K(S\{u\}).B::cyt + C.D\{i\}::cyt + A::cyt \Rightarrow K(S\{p\}.B.C::cyt + D\{a\}.A::cyt + H\{u\}::cyt$}
\end{example}

\noindent\rule{\textwidth}{1pt}

\section{Semantic function}

Semantic function $\mathtt{F}$ gives semantical meaning to rules and objects written in BCSL syntax. It is defined recursively according to context given as argument:

\begin{center}
\begin{tabular}{ r c l}
$\mathtt{F} \llbracket ~\mu~ \rrbracket$ & = &
		$\begin{cases}
		\mathtt{A}(\mu, \epsilon) ~\ldots~ \mu \in \Sigma_{A}\\
		\mathtt{T}(\mu, \emptyset) ~\ldots~ \mu \in \Sigma_{T}\\
		\end{cases}
		$\\
 & & \\
$\mathtt{F} \llbracket ~\mu\{s\}~ \rrbracket$ & = & $\mathtt{A}(\mu, s)$\\
 & & \\
$\mathtt{F} \llbracket ~\mu(\mathtt{a}_1, \ldots, \mathtt{a}_n)~ \rrbracket$ & = &
$\mathtt{T}(\mu, \{ \mathtt{F} \llbracket \mathtt{a}_1 \rrbracket, \ldots, \mathtt{F} \llbracket \mathtt{a}_n \rrbracket \})$\\
 & & \\
$\mathtt{F} \llbracket ~\alpha_1~.~\ldots~.~\alpha_n :: c~ \rrbracket$ & = &
$\mathtt{X}((\mathtt{F} \llbracket ~\alpha_1~ \rrbracket, \ldots, \mathtt{F} \llbracket ~\alpha_n~ \rrbracket), c)$\\
 & & \\
$\mathtt{F} \llbracket ~\Gamma_1 + \ldots + \Gamma_n \Rightarrow \Gamma_{n+1} + \ldots + \Gamma_m~ \rrbracket$ & = &
$(\chi, \omega, \mathtt{I}, \mathtt{map}, \mathtt{Indices})$~ such that:\\
\end{tabular}
\end{center}

\begin{center}
\begin{itemize}[leftmargin=1.5cm]
\item $\chi = (\mathtt{F} \llbracket ~\Gamma_1~ \rrbracket, \ldots, \mathtt{F} \llbracket ~\Gamma_n~ \rrbracket, \mathtt{F} \llbracket ~\Gamma_{n+1}~ \rrbracket, \ldots, \mathtt{F} \llbracket ~\Gamma_m~ \rrbracket)$,
\item $\omega = \bullet_{\mathtt{X} \in \chi} (\mu(\mathtt{X}))$ !!!
\item $\mathtt{I} = n$
\item $\mathtt{map} = (J_1, \ldots, J_m): J_i = \sum_{y=1}^{i} | \mu(\chi_i) |$
\item \begin{tabular}{c l}

& $\{~ (i,j) ~|~ i \in [1, \mathtt{map}(\mathtt{I})] \wedge j \in [\mathtt{map}(\mathtt{I}), |\omega|] \wedge |i-j| \equiv \mathtt{map}(\mathtt{I})~\} ~\cup$ \\
$\mathtt{Indices} =$ & $\{~ (i, -) ~|~ i \in [k, \mathtt{map}(\mathtt{I})] \wedge k = |~ \{ \mathtt{map}(\mathtt{I}) + 1, \ldots, | \alpha | \} ~| ~\} ~\cup$\\
 & $ \{~ (-, j) ~|~ j \in [k, |\alpha|] \wedge k = 2 * \mathtt{map}(\mathtt{I}) ~\}$ (!!! ordered set)
\end{tabular}

\end{itemize}
\end{center}

Please note by applying semantic function $\mathtt{F}$ on examples from Example~\ref{example:syntax}, we obtain agents (resp. complex or rule) from appropriate referenced examples.

\noindent\rule{\textwidth}{1pt}

\section{Additional definitions}

\begin{definition}{Tuples concatenation}

Let $T = (\tau_1, \ldots, \tau_n)$ be sequence of tuples where $\tau_i = (x_1^i, \ldots, x_m^i)$. Concatenation of sequence of tuples $\bullet_{\tau \in T} (\tau)$ is defined as:

\begin{center}
$\bullet_{\tau \in T} (\tau) = (x_1^1, \ldots, x_m^1, x_1^2, \ldots, x_l^2, \ldots \ldots, x_1^n, \ldots, x_k^n)$.
\end{center}
\end{definition}

\begin{definition}{Difference of partial compositions}

Let $\gamma, \gamma'$ be two partial compositions. We define difference of these two sets on names of its atomic agents as following:

\begin{center}
$\gamma \ominus \gamma' \Leftrightarrow \{ \mu_\alpha ~|~ \exists \mathtt{A}: (\mathtt{A} \in \gamma \wedge \mu(\mathtt{A}) \equiv \mu_\alpha ) \wedge \not\exists \mathtt{A}': (\mathtt{A}' \in \gamma' \wedge \mu(\mathtt{A}') \equiv \mu_\alpha)\}$
\end{center}
\end{definition}

\begin{definition}{Difference of partial composition and structure signature}

Let $\mathtt{T}$ be a structure agent. Next, let $\gamma(\mathtt{T})$ be its partial composition and $\Sigma_T(\mathtt{T})$ appropriate signature. We define difference of these two sets on names of atomic agents as following:

\begin{center}
$\gamma(\mathtt{T}) \ominus \Sigma_T(\mathtt{T}) \Leftrightarrow \{ \mu_\Sigma ~|~ \mu_\Sigma \in \Sigma_T(\mathtt{T}) \wedge \not\exists \mathtt{A}: (\mathtt{A} \in \gamma(\mathtt{T}) \wedge \mu(\mathtt{A}) \equiv \mu_\Sigma)\}$
\end{center}
\end{definition}

\noindent\rule{\textwidth}{2pt}

\section{Ground forms}

Ground form represents appending context to agents according to signature (we assume signature is always available and therefor we omit it in ground form function arguments).

Objects are transformed to their ground forms by ground form function $\mathcal{G}$ defined as following:

\begin{center}
\begin{tabular}{ r c l }
Object & basic form & ground form \\
\hline
 & & \\
$\mathtt{A}$ & $\mathcal{G}((\mu, \epsilon))$ & $\{~ (\mu, \delta) ~|~ \delta \in \Sigma_A(\mu) ~\}$\\
 & $\mathcal{G}((\mu, \delta))$ & $\{~(\mu, \delta) ~\}$\\
 & & \\
 \hline
 & & \\
$\mathtt{T}$ & $\mathcal{G}((\mu, \gamma))$ & $\{~ (\mu, \gamma_\alpha) ~|~ \gamma_\alpha \equiv \gamma \cup \gamma_\Sigma \wedge \gamma_\Sigma = \{~ \mathtt{A}_1, \ldots, \mathtt{A}_n ~\} \wedge$\\
 & & $\wedge~ \mathtt{A}_i \in \mathcal{G}((\mu, \emptyset)) \wedge \mu \in \gamma \ominus \Sigma_T(\mathtt{T}) ~\}$ \\
 & & \\
 \hline
 & & \\
pair of & $\mathcal{G}((\beta, \beta'))$ & $\{~ (\beta_\alpha, \beta_\alpha') ~|~ \beta_\alpha \in \mathcal{G}(\beta) \wedge \beta'_\alpha \in \mathcal{G}(\beta') \wedge$\\
agents & & $\wedge~ \gamma(\beta_\alpha) \ominus \gamma(\beta) \equiv \gamma(\beta'_\alpha) \ominus \gamma(\beta') ~\} $ \\
 & & \\
 \hline
 & & \\
$\mathtt{R}$ & $\mathcal{G}(\mathtt{R})$ & $ \bigotimes \mathcal{G}(\mathtt{Indices})^* $ \\
 & $\mathcal{G}(\mathtt{Indices})$ & $(~\mathcal{G}(\omega(i), \omega(j)) ~|~ (i,j) \in \mathtt{Indices})$\\
 
\end{tabular}
\end{center}

where $\bigotimes \mathcal{G}(\mathtt{Indices})^*$ means cartesian product of sets inside of set $\mathcal{G}(\mathtt{Indices})$.

\noindent Next, we need reassemble produced tuples by concatenation.

\begin{center}
$\mathtt{reassemble}(\mathcal{G}(\mathtt{R})) = \{~  T ~|~ T = \bullet_{\tau \in \iota} (\tau) \wedge \iota \in \mathcal{G}(\mathtt{R}) ~\}$
\end{center}

\noindent\rule{\textwidth}{2pt}

\section{Generating reactions}
\label{Generating reactions}

Given a rule $\mathtt{R}$, we can produce appropriate reactions using ground forms and reassembly:

\begin{tabular}{ ll }
$\mathtt{Reactions} =$ & $\{~ \mathtt{r}(\mathtt{seq}, \mathtt{I}) ~|~ T \in \mathtt{reassemble}(\mathcal{G}(\mathtt{R})) ~\wedge$\\

& $seq = (\mathtt{X}(T(1, \ldots, i_1), \mathtt{com}(\chi(i_1))), \mathtt{X}(T(i_1 + 1, \ldots, i_2), \mathtt{com}(\chi(i_2))), \ldots,$\\

& $\mathtt{X}(T(i_{n-1}+1, \ldots, i_n), \mathtt{com}(\chi(i_n)))) ~\wedge$\\

& $ \mathtt{I} \in \mathtt{R} \wedge \chi \in \mathtt{R} ~\}$
\end{tabular}

\noindent\rule{\textwidth}{1pt}

\section{Vectorized model}

\begin{definition}{Vectorized model}

Vectorized model $\mathcal{M}$ is a triple $(\mathcal{R}, \nu, \pi)$ such that $\mathcal{R} \subseteq \mathbb{Z}^n$ is set of vector reactions, $\nu \in \mathbb{N}_0^n$ is initial vector state, and $\pi \in \chi^n$ is vector of reference complexes for some $n \in \mathbb{N}$.
\end{definition}

\begin{notation}
We denote $N(\nu) ~\emph{iff}~ \forall i \in \nu: i \in \mathbb{N}_0$.
\end{notation}

\begin{definition}{Vector reaction application}

Vector reaction application $\rho \subseteq \mathbb{N}_0^n \times \mathbb{Z}^n \times \mathbb{N}_0^n$ such that 
\begin{center}
$(\upsilon, \mathtt{r}, \mathtt{u}) \in \rho$ $\Leftrightarrow$ $u = \upsilon + \mathtt{r} \wedge N(u)$.
\end{center}
\end{definition}

By iterative application of reaction application we obtain LTS (which can be bounded by a global bound).

\noindent\rule{\textwidth}{1pt}

\section{Production of model}

\begin{definition}{Reference vector}

Reference vector $\pi$ is a tuple of all possible unique complexes constructed from reactions as following:

\begin{center}
$\mathcal{U}(\mathtt{Reactions}) = \{~ \mathtt{X} ~|~ \mathtt{X} \in \mathtt{seq}(\mathtt{r}) \wedge \mathtt{r} \in \mathtt{Reactions} ~\} $.
\end{center}

Then, we just choose a random order of the set.
\end{definition}

\begin{definition}{State to vector translation}

Translation of state $\mathtt{S}$ to vector is defined as following:
\begin{center}
$\lambda(\mathtt{S}, \pi) = (a_1, a_2, \ldots, a_n)$ such that 
$a_i$ =
	$\begin{cases}
	|\mathtt{S}(\pi_i)| & \ldots~~~~ \pi_i \in \mathtt{S}\\
	0 & \ldots~~~~ \pi_i \not\in \mathtt{S}\\
	\end{cases}
	$
\end{center}
\end{definition}

Let $(\mathtt{S}, \mathtt{R}, \Sigma)$ be a rule-based model. In order to construct vectorized model $\mathcal{M} = (\nu, \mathcal{R}, \pi)$, we need to do following steps:

\begin{enumerate}
\item generate $\mathtt{Reactions}$ as defined in section~\ref{Generating reactions},
\item construct reference vector $\pi$ from $\mathtt{Reactions}$,
\item create $\nu$ as $\lambda(\mathtt{S}, \pi)$,
\item construct set of vector reactions $\mathcal{R}$ from set of $\mathtt{Reactions}$, where individual vector reaction $\vv{r}$ are created from $\mathtt{r}$ as following:

\begin{center}
$\vv{r} = \lambda(RHS(\mathtt{r}), \pi) - \lambda(LHS(\mathtt{r}), \pi)$
\end{center}

\item resulting triple $(\nu, \mathcal{R}, \pi)$ is desired model $\mathcal{M}$.
\end{enumerate}

\end{document}